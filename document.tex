\documentclass[12pt]{article}
\usepackage{listings}
\usepackage[utf8]{inputenc}
\usepackage[none]{hyphenat}

\title{Meu VIM Tutor}
\author{Leirdan}
\date{\today}

\begin{document}

\maketitle

\section{Introdução}
	\subsection{O que é o VIM?}
	O VIM é um editor de texto amplamente utilizado ao redor do mundo. Criado como \textbf{VI Improved}, oferece ampla gama de configurações e comandos especiais, tornando-o um editor extremamente personalizável. Oferece suporte a mais de 500 extensões de arquivos, macros, buscas por expressões regulares e outros. 
	\subsection{Como instalar}
	Consulte de download: https://www.vim.org/download.php. Para distribuições GNU/Linux, verifique no repositório de pacotes se há um pacote já criado por meio de, por exemplo:
	\begin{lstlisting}
		sudo apt-get install vim
	\end{lstlisting}
	\subsection{Modos de operação}
	O diferencial do Vim são seus modos, que permitem ao usuário realizar operações rapidamente. São eles:
		\subsubsection{Modo Normal}									
		O modo mais utilizado, tem a função de \emph{formatar, copiar, mover cursor, etc.} Nele, podemos mover o cursor com as teclas \emph{h} (esquerda), \emph{j} (baixo), \emph{k} (cima), \emph{l} (direita).
		\begin{itemize}
			\item Se digitar \textbf{$n$j}, desce \emph{n} linhas;
			\item Se digitar \textbf{$n$l}, avança \emph{n} caracteres à direita;
			\item Se digitar \textbf{w}, move para o começo da palavra;
			\item Se digitar \textbf{b}, move para o começo da palavra anterior;
			\item Se digitar \textbf{e}, move para o final da palavra;
			\item Se digitar \textbf{0}, move para o começo da linha;
			\item Se digitar \textbf{\$}, move para o final da linha.	
		\end{itemize} 
		Além disso, temos os seguintes comandos:
		\begin{itemize}
			\item Se digitar \textbf{r} em cima de um caractere, pode substituí-lo por outro;
			\item Se digitar \textbf{x} em cima de um caractere, apaga-o;
			\item Se digitar \textbf{u}, desfaz todas as suas alterações e restaura o documento até a última vez que entrou no \emph{normal mode};
			\item Para refazer mudanças causadas pelo \textbf{u}, insira \textbf{ctrl + r}.
		\end{itemize}
		\subsubsection{Modo de Inserção}
		O modo de inserção é autoexplicativo: faz o VIM comportar-se como um editor de texto simples, permitindo escrever e editar arquivos. Podemos entrar neste modo das seguintes formas:
		\begin{itemize}
			\item No modo normal, teclando \textbf{i} começa a editar de onde o cursor está parado;
			\begin{itemize}
				\item Teclando \textbf{I} começa a editar do começo da linha;
			\end{itemize}
			\item No modo normal, teclando \textbf{a} começa a editar um caractere após o caractere atual (muito útil para inserir textos após o fim de uma frase);
			\begin{itemize}
				\item Teclando \textbf{A} começa a editar a partir do final da linha;
			\end{itemize}
			\item No modo normal, teclando \textbf{o} insere uma nova linha abaixo da atual e edita a partir do começo desta nova.
			\begin{itemize}
				\item Teclando \textbf{O} insere uma nova linha acima da atual e edita a partir do começo desta nova.
			\end{itemize}
		\end{itemize}
		Para sair do modo de inserção, tecle \emph{esc}.
		\subsubsection{Modo Visual}
		O modo visual é usado para fazer seleções de textos (como um \emph{mouse} faria) e recortar, copiar, excluir, entre outras opções. Para entrar no modo visual temos as opções:
		\begin{itemize}
			\item Pressionar \textbf{v} faz o cursor iniciar uma seleção do caractere onde estava;
			\item Pressionar \textbf{V} permite fazer a seleção de uma linha inteira ou mais;
			\item Pressionar \textbf{ctrl + v} permite selecionar uma região retangular no texto e ao longo das linhas que quiser.  
		\end{itemize}
		\subsubsection{Modo de Comando}
		Um dos modos mais poderosos do VIM, provê funcionalidades que o modo normal não consegue, como a execução de comandos específicos e externos ao editor. Para entrar neste modo, basta estar no modo normal e teclar ":". 
	\newpage

\section{Editando}
	\subsection{Abrir arquivo}
	Para abrir um arquivo no editor VIM, basta digitar:
	\begin{lstlisting}
		vim [nome do arquivo]
	\end{lstlisting}
	Se desejar abrir o arquivo em uma linha específica, pode-se usar:
	\begin{lstlisting}
		vim +42 [nome do arquivo]
	\end{lstlisting}
	para abrir na linha 42. \\
	Caso esteja dentro de um arquivo e deseje abrir outro, pode-se inserir o seguinte comando no modo de comando:
	\begin{lstlisting}
		:e [nome do arquivo]
	\end{lstlisting} 
	\subsection{Salvar/sair do arquivo}
	Para salvar um arquivo no editor VIM, basta estar no modo de comando e digitar \textbf{:w}. \\
	Para sair do arquivo, basta digitar no modo de comando \textbf{:q}. Se tiverem mudanças não salvas, pode-se digitar \textbf{:q} para sair de maneira forçada; caso queira salvar e sair, pode-se digitar \textbf{:wq!}. 
	\subsection{Copiar textos}
	A letra \textbf{y} está associada ao ato de \emph{copiar textos} no editor VIM. Assim, temos as seguintes possibilidades:
	\begin{itemize}
		\item Pressionar \textbf{yy} ou \textbf{Y} copia a linha atual inteira;
		\item Pressionar \textbf{ye} copia do cursor ao fim da palavra;
		\item Pressionar \textbf{yb} copia do cursor ao começo da palavra;
		\item Note que as letras \textbf{e} e \textbf{b} que acompanham o comando \textbf{y} são as teclas vistas na seção do \emph{normal mode}.
	\end{itemize}
\end{document}