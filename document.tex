\documentclass[12pt]{article}
\usepackage{listings}
\usepackage[utf8]{inputenc}
\usepackage[none]{hyphenat}

\title{Meu VIM Tutor}
\author{Leirdan}
\date{\today}

\begin{document}

\maketitle

\section{Introdução}
	\subsection{O que é o VIM?}
	O VIM é um editor de texto amplamente utilizado ao redor do mundo. Criado como \textbf{VI Improved}, oferece ampla gama de configurações e comandos especiais, tornando-o um editor extremamente personalizável. Oferece suporte a mais de 500 extensões de arquivos, macros, buscas por expressões regulares e outros. 
	\subsection{Como instalar}
	Consulte de download: https://www.vim.org/download.php. Para distribuições GNU/Linux, verifique no repositório de pacotes se há um pacote já criado por meio de, por exemplo:
	\begin{lstlisting}
		sudo apt-get install vim
	\end{lstlisting}

\end{document}
